Dies ist die erste Version der Storyplanung, diese wird sich wahrscheinlich im Laufe der Zeit zumindest teilweise ändern. Intern arbeiten wir mit Github Issues, welche für das Projekt auf der \fnurl{Github Issues Seite}{https://github.com/swp14-gsb/gsb/issues} zu finden sind und eher den aktuellen Entwicklungsstand darstellen.

\begin{release}{22.04.2014}
Christoph steht aufgrund Urlaubs nicht für dieses Releasebündels zur Verfügung.

\story{Auftrennen der Controllerlogik}{Niemand}{4}{}%
Es geht beim Auftrennen der Controllerlogik darum, sich Gedanken zu machen, inwiefern ein Aufbrechen der Bestehenden Controller Sinn ergibt und wie dieses geschehen kann, um die zukünftige Arbeit zu erleichtern. Zu diesem Zeitpunkt scheinen folgende verschachtelte Controller (von Außen nach Innen) sinnvoll:
\begin{enumerate}
\item \textbf{App:} Kontrolliert die Anwendung, hier werden allgemeine Einstellungen vorgenommen und Dinge wie eine die Übersetzungen von GSBL zu JSON / JSON zu SPARQL in Gang gesetzt
\item\textbf{SubjectCollection:} Die Sammlung aller Subjekte, hier wird das hinzufügen und entfernen von Subjekten erledigt.
\item\textbf{SubjectInstance:} Ein Individuelles Subjekt, hier werden alle Funktionen vorgenommen, die das einzelne Subjekt betreffen
\item\textbf{PropertyCollection:} Analog zur SubjectCollection, bloß für die Eigenschaften
\item\textbf{PropertyInstance:} Eine Individuelle Eigenschaft, eventuell noch aufspaltbar (oder zusätzliche Ebene), um Datentypeigenschaften und Objekteigenschaften zu unterscheiden
\end{enumerate}

\Hack{\enlargethispage{2.5\baselineskip}}
\story{availableClassService zum endpointService erweitern}{Niemand}{2}{}%
Einen Service Schaffen, der mehrere Funktionen hat, um Metadaten aus dem Endpoint auszulesen, wie zum Beispiel:
\begin{itemize}
\item getAvailableClasses()
\item getProperties(uri) 
\item getInverseProperties(uri)
\end{itemize}

\story{Übersetzungsfunktion als eigener Service }{Niemand}{3}{}%
Analog zum endpointService sollte ein translationService geschaffen werden, der die Übersetzung zu JSON und zu SPARQL bereitstellt, mit Funktionen wie:
\begin{itemize}
\item translateToJSON(gsbl)
\item translateToSPARQL(json)
\end{itemize}

\story{Drei-Schritte-Layout \& Draggable}{Niemand}{2}{}%
Mergen des bereits existierenden Draggable-Branches in den Master. Des weiteren Umsetzung eines Dreigeteilten Layouts, zum Beispiel durch Tabs, in dem jeder Tab einen logischen Schritt des Workflows des GSB entspricht:
\begin{enumerate}
\item Workspace: Erstellung des GSBL-Wortes
\item Queries: Übersetzte JSON \& SPARQL-Anfragen
\item Results: Ergebnisse einer SPARQL-Anfrage
\end{enumerate}
\end{release}


\begin{release}{28.04.2014}
Christoph steht aufgrund Urlaubs nicht für dieses Releasebündels zur Verfügung.

\story{Endpunktanbindung}{Niemand}{4}{}%
Anbinden eines Endpunkts in den endpointService (durch erstellen passender Konfigurationen) durch geeignete SPARQL-Queries, sodass auf echten Daten eines Endpunkts gesucht werden kann. Vorzugsweise sollte dies an einem \fnurl{ERM Endpunkt}{http://aksw.org/Projects/ERM.html}  geschehen. Falls dies nicht möglich ist, muss auf die \fnurl{dbpedia}{http://dbpedia.org} zurückgegriffen werden.\\
\textbf{Achtung:} Es sollten auch die Eigenschaften der Überklassen ausgelesen werden (Stichwort: Hierarchien in Subjektklassen).

\story{Objekteigenschaften: Inverse Beziehung}{Niemand}{4}{}%
Umsetzung der Inversen Beziehung (\textit{is of}).\\
Also zum Beispiel: \glqq Universität \textit{is} Studienort \textit{of} Person\grqq

\story{Datentypeigenschaft: Zahlenwerte}{Niemand}{6}{}%
Arithmetik, Vergleiche und Übersetzung zu SPARQL von\\
\textit{xsd:integer}, \textit{xsd:decimal}, \textit{xsd:float} und \textit{xsd:double}.

\story{Datentypeigenschaft: Strings}{Niemand}{6}{}%
Vergleiche (equals, contains, starts|endsWith,  Regex) und Langparameter von \textit{xsd:string}, sowie Übersetzung zu SPARQL
\end{release}


\begin{release}{05.05.2014}

\story{Datentypeigenschaft: Daten \& Uhrzeiten}{Niemand}{6}{}%
Vergleiche: Datums- (\textit{xsd:dateTime}, \textit{xsd:date}) und Zeiträume (\textit{xsd:time}), sowie Übersetzung zu SPARQL.

\story{Optionale Properties}{Niemand}{3}{}%
Möglichkeit schaffen eine Eigenschaft als optional zu deklarieren.

\story{Sonderfall: Anonyme Klasse}{Niemand}{3}{}%
Eine Subjektklasse, die \textit{owl:Thing} repräsentiert und mit der alle Eigenschaften der Ontologie durchsucht werden können.

\story{Testcases 1}{Niemand}{4}{}%
Tests für die bis zum Start des Releasebündel R3 fertigen Klassen. (Controller \& Services.)
\end{release}


\begin{release}{12.05.2014}

\story{JSON speichern \& Laden}{Niemand}{5}{}%
Speichern und Laden eines GSBL-Wortes im JSON-Formates.

\story{Dokumentation Code-Ebene}{Niemand}{3}{}%
Komplette Dokumentation auf Codeebene gemäß unseres Qualitätskonzepts.

\story{Legende \& Benutzerhandbuch}{Niemand}{4}{}%
Erstellung einer Legende in der Oberfläche, Erstellung eines Benutzerhandbuchs in HTML.

\story{Testcases 2}{Niemand}{3}{}%
Ggf. Anpassung Tests und neue Tests für die neu erstellten Klassen.
\end{release}


\begin{release}{19.05.2014}

\story{Platzhalter}{Niemand}{??}{X}%
Für die letzte Woche ist vor allem vorgesehen eventuelle Mängel, nicht geschafftes und Kannziele umzusetzen.
\end{release}


%%% Local Variables: 
%%% mode: latex
%%% TeX-master: "main"
%%% End: 
