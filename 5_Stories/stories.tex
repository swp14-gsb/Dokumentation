\begin{bundle}{22.04.2014}
\release{Tolles feature}{Siggi}{3}%
Es geht beim Auftrennen der Controllerlogik darum, sich Gedanken zu machen, inwiefern ein Aufbrechen der Bestehenden Controller Sinn ergibt und wie dieses geschehen kann, um die zukünftige Arbeit zu erleichtern. Zu diesem Zeitpunkt scheinen folgende ve


\release{doofes zeug}{jemand}{80}%
Es geht beim Auftrennen der Controllerlogik darum, sich Gedanken zu machen, inwiefern ein Aufbrechen der Bestehenden Controller Sinn ergibt und wie dieses geschehen kann, um die zukünftige Arbeit zu erleichtern. Zu diesem Zeitpunkt scheinen folgende ve
\end{bundle}


\begin{bundle}{28.04.2014}
\release{Tolles feature}{Siggi}{3}%
Es geht beim Auftrennen der Controllerlogik darum, sich Gedanken zu machen, inwiefern ein Aufbrechen der Bestehenden Controller Sinn ergibt und wie dieses geschehen kann, um die zukünftige Arbeit zu erleichtern. Zu diesem Zeitpunkt scheinen folgende ve


\release{doofes zeug}{jemand}{80}%
Es geht beim Auftrennen der Controllerlogik darum, sich Gedanken zu machen, inwiefern ein Aufbrechen der Bestehenden Controller Sinn ergibt und wie dieses geschehen kann, um die zukünftige Arbeit zu erleichtern. Zu diesem Zeitpunkt scheinen folgende ve
\end{bundle}


\begin{bundle}{05.05.2014}

\end{bundle}


\begin{bundle}{12.05.2014}

\end{bundle}


\begin{bundle}{19.05.2014}

\end{bundle}


%%% Local Variables: 
%%% mode: latex
%%% TeX-master: "main"
%%% End: 
