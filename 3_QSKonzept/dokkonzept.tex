
\section{Dokumentationskonzept}

\subsection{Interne Dokumentation}

\subsubsection{Sprache}

Bezeichner und Kommentare sind auf Englisch zu verfassen.
Umlaute und Eszett sind im Quelltext zu vermeiden.

\subsubsection{Coding Standard}

Das Projekt folgt den Google Style Guides für JavaScript
(\url{http://google-styleguide.googlecode.com/svn/trunk/javascriptguide.xml})
und HTML (\url{http://google-styleguide.googlecode.com/svn/trunk/javascriptguide.xml}).

Darüber hinausgehende Konventionen, sowie ausgewählter der sich daraus ergebenden Quelltextkonventionen sind im Folgenden aufgeführt.

\paragraph{Einrückung (JavaScript, HTML)}

Eine Einrückungsebene wird durch zwei Leerzeichen ausgezeichnet.
Tabulatoren werden nicht zur Einrückung verwendet.

\paragraph{Sprechende Variablennamen (JavaScript)}

Der Name eine Variable soll ihren Verwendungszweck erkennen lassen.

\paragraph{Variablendeklaration (JavaScript)}

Variablen werden zu Beginn ihres Gültigkeitsraums deklariert,
z.\,B. am Beginn einer Funktion.

\paragraph{Geschweifte Klammern öffnen auf gleicher Zeile (JavaScript)}

Gruppierungen durch geschweifte Klammern werden auf der Zeile
geöffnet, auf der die Art der Gruppe definiert wird.
Z.\,B.

if (something) {
  // ...
} else {
  // ...
}

an Stelle von

if (something) 
{
  // ...
} else 
{
  // ...
}

\paragraph{Bezeichner (JavaScript)}

Funktionen und Variablen beginnen mit Kleinbuchstaben.
Klassennamen beginnen mit Großbuchstaben.
Die Trennung von Worten erfolgt durch Binnenmajuskel (auch als camel
case bezeichnet).
Konstante Variablen werden komplett mit Großbuchstaben getrennt durch
Unterstrich benannt.
Optionale Funktionsargumente beginnen mit ``opt\_''.

Beispiele:
\begin{itemize}
\item functionNamesLikeThis
\item variableNamesLikeThis
\item ClassNamesLikeThis
\item EnumNamesLikeThis
\item methodNamesLikeThis
\item CONSTANT\_VALUES\_LIKE\_THIS
\item foo.namespaceNamesLikeThis.bar
\item filenameslikethis.js
\end{itemize}

\paragraph{Funktions- und Klassenlänge (JavaScript)}

Die Länge einer einzelnen Funktion ist auf 15 Zeilen beschränkt, die
einer Klasse auf 120.

\paragraph{Apostroph gegenüber Anführungszeichen bevorzugen}

Soweit möglich wird ' bevorzugt gegenüber " verwendet.

\paragraph{Nur Kleinbuchstaben (HTML)}

Der gesamte Quelltext ist in Kleinbuchstaben zu verfassen.
Ausgenommen sind lediglich Kommentare und String-Attribut-Werte.

\subsubsection{Dokumentationsgenerator}

Zum generieren der Quelltext-Dokumentation für JavaScript wird der
Dokumentations-Prozessor JSDoc3 verwendet
(\url{https://github.com/jsdoc3/jsdoc}).
JSDoc bezeichnet nicht nur die Software zum generieren der
Dokumentation sondern auch die gleichnamige Metasprache zum
Kommentieren von JavaScript.
Da der in diesem Projekt verwendete Google JavaScript Style Guide die
Verwendung von JSDoc vorschreibt liegt die Dokumentations-Gernerierung
mittels JSDoc3 nahe.

\subsubsection{Qualitätssicherung}

Als Werkzeug zur Qualitätssicherung kommt JSLint zum Einsatz.
JSLint überprüft JavaScript-Quelltext auf syntaktische Fehler und
stilistische Schwächen und ist sowohl als Online-Tool
(\url{http://www.jslint.com}) als auch als Kommandozeilen-Tool
verfügbar (\url{http://www.jshint.com/install/}).
JSLint bitetet eine Reihe von einstellabren Optionen, beispielsweise
zur Festlegung der gewünschten Einrückung und soll als Hilfsmittel
verstanden werden, die selbst auferlegten Quelltextkonventionen einzuhalten.

\subsection{Produktions-Dokumentation}

Die Arbeit am Projekt wird sowohl durch die wöchentliche
Aufwandserfassung als auch durch die Verwendung des
Versionskontrollsystems Git dokumentiert.

\subsection{Entwurfs-Dokumentation}

Die Entwurfs-Dokumentation dient sowohl den Projektteilnehmern zum
Festhalten getroffener Struktur- und Design-Entscheidungen als auch
Aussenstehenden als Einführung in das Projekt.
Die Entwurfs-Dokumentation ist auch als Technische Dokumentation der
grundlegenden Systemarchitektur zu verstehen.

Sie umfasst eine Beschreibung der Zielstellung und Motivation des
Projekts.
Sie beschreibt den äußerlichen Funktionsumfang des Systems, also die
Perspektive des Anwenders.
Weiter werden die angewendeten Struktur- und Entwurfsprinzipien,
sowohl des Gesamtsystems als auch von Teilsystemen, beschrieben,
erläutert und begründet. Dies dient als Einstiegshilfe in das
Softwaresystem, bevor sich den Details der Umsetzung zugewendet wird.
Ebenfalls dokumentiert wird das Datenmodell und das Testkonzept der
Software.
Ein Glossar ist ebenfalls Bestandteil der Entwurfsdokumentation.


%%% Local Variables: 
%%% mode: latex
%%% TeX-master: "main"
%%% End: 
