\section{Dokumentationskonzept}

Das bisherige Dokumentationskonzept behält weiterhin seine Gültigkeit.

\section{Testkonzept }

Ein Projekt diesen Umfangs bedingt auch ein hohes Maß an Komplexität -- was wiederum unvermeidlich zu Fehlern im Entwicklungsprozess führt. Somit ist es unabdingbar, einem entgegenwirkenden Konzept für ein kontinuierliches Testen zu folgen, welches dazu dienen soll, Fehler frühzeitig und systematisch zu identifizieren und effektiv zu entfernen. Dafür werden unterschiedliche Testphasen unterschieden.

\subsection*{Komponententest}
Komponententests (oder auch Unit-Test) betrachten einzelne Komponenten der Anwendung, also Klassen bzw. Module, isoliert vom Gesamtkonstrukt. Durch diese losgelöste Bearbeitung wird ein externer Einfluss so weit wie möglich ausgeschlossen und sichergestellt, dass zunächst interne Fehler bereinigt werden. Diese Komponententest (bzw. Unit-Tests) sollten also möglichst unmittelbar nach der Fertigstellung der Komponente durchgeführt werden. Dafür bietet das für den GSB genutzte Framework AngularJS dabei bereits eigene, potente Werkzeuge an\footnote{\url{http://docs.angularjs.org/guide/unit-testing}}. Diese erlauben losgelößte Tests einzelner Funktionen, ohne etwa von der Antwort eines XMLHttpRequest anhängig sein und diesen stattdessen zu simulieren. Für den GSB betreffen diese Tests besonders Controller (siehe dazu auch\footnote{\url{http://docs.angularjs.org/guide/controller}}) und Services und sollten entweder nach deren Fertigstellung geschrieben werden, oder bereits davor, um den entsprechenden Code “entgegenkommend” zu entwickeln. Dies ist besonders relevant für Features, die es nicht in den endügltigen Release schaffen, später jedoch eventuell von einer anderen Partei fertiggestellt werden. Der bereits bestehende Test könnte dann als eine Art Leitfaden bereitgestellt werden.%

\subsection*{Integrationstest}
Ein Integrationstest betrachtet vor allem die Interaktion mehrerer Komponenten untereinander, also deren Funktionsschnittstellen, bzw. Kommunikation untereinander. Ein solcher Test ist damit besonders dann sinnvoll, wenn eine einzelne Komponente fertiggestellt wurde, und es nun auf deren Einbettung in bereits vorhandenen Anwendungsstrukturen ankommt. Somit sollten fertige Komponenten stets auf reibungslose Integration geprüft werden. Auch dafür können die AngularJS-eigenen Tools genutzt werden.

\subsection*{Systemtest}
Beim Systemtest schließlich werden alle Komponenten der Anwendung zusammengeführt und auf Funktionsfähigkeit geprüft. Als Spezialfall des Systemtest kann man hierbei den Abnahmetest ansehen, der quasi den finalen Test vor der Abnahme des Programms darstellt.
Für Systemtests bietet Protractor ein Testframework für AngularJS-Applikationen\footnote{\url{https://github.com/angular/protractor} }. Dabei kann die Anwendung direkt im Browser getestet und mit ihr aus der Sicht eines Users interagiert werden.%

Es soll außerdem, als Variation des Systemtests, ein besonderes Augenmerk auf Nutzungstests durch Projekt-fremde User gelegt werden. Dadurch kann Feedback zu den für den GSB essentiellen Aspekten Benutzbarkeit und Intuitivität gewonnen werden, von Laien, die ihrerseits keine weitreichenden Erfahrungen mit SPARQL haben und sich dadurch mit der Zielgruppe des Projekts decken. Auch das Aufdecken möglicher Ausfälle bei falscher Benutzung sind denkbar.
Im Besonderen betrifft dies die Mitarbeiter der Bibliotheca Albertina, welche als mögliche Abnehmer des GSB bereits feststehen. Deren Rückmeldungen sollen insofern stark gewichtet werden, da sie besonders “praxisnahe” Probleme aufzeigen und Verbesserungsvorschläge liefern könnten. Zu diesem Zweck ist auch ein regelmäßiges Treffen geplant.

\section{Organisatorisches}

Die bereits etablierten, wöchentlichen Gruppentreffen sollen weiterhin stattfinden, jedoch abhängig vom generellen Bedarf für eine gemeinsame Besprechung, bzw. für eine Einbeziehung der Betreuer. Sie dienen in dieser letzten Phase hauptsächlich der Besprechung auftretender Probleme, Fragen und bearbeiteter Ergebnisse. Innerhalb der Gruppe werden Absprachen weiterhin über EMail, WhatsApp und Skype erfolgen, und Treffen untereinander spontan getätigt.
Außerdem sollen, in regelmäßigen Abständen mehrerer Wochen, Absprachen mit den Mitarbeitern der Bibliotheca Albertina getroffen werden, welche als potentielle Abnehmer des GSB bereits feststehen. Dabei können der Projektfortschritt demonstriert, und Rückmeldungen gesammelt werden, welche dann möglicherweise noch in die weitere Implementierung miteinbezogen werden.

Es soll auch weiterhin ein einheitlicher Kenntnisstand aller Mitglieder angestrebt werden und die einzelnen Teilarbeiten auch während der Bearbeitung des Einzelnen, bzw. innerhalb der Programmierpaare, möglichst einsehbar zur Verfügung stehen. Dies soll besonders der gegenseitigen Unterstützung dienen. Es ist natürlich aber unabdingbar, bei Problemen mit der zugeteilten Aufgabe, Zeitmangel, etc. rechtzeitig Bescheid zu geben, sodass entsprechende Hilfe\,/\,Anpassungen der Aufgabenverteilung, bzw. der Storyplanung möglich sind.
Besprochenes und Bearbeitetes wird in Protokollen, bzw. besonders im Bezug auf die eigentliche Anwendung mithilfe des Versionsverwaltungssystems git festgehalten.

Alle Projektteilnehmer sollen sich an die vereinbarten Dokumentationsvorgaben halten. Dies sollte von den entsprechenden Verantwortlichen stichprobenartig geprüft werden, besonders vor Abgabe der Releasebündel. Die wöchentliche Aufwandserfassung ist von jedem Mitglied im entsprechenden GoogleDoc rechtzeitig zu aktualisieren.

%%% Local Variables: 
%%% mode: latex
%%% TeX-master: "main"
%%% End: 
