%\def\dokument{Risikoanalyse} % Dokumentbezeichnung in Kopfzeile
%\def\verantwortlicher{} % Dokumentverantwortlicher in Kopfzeile
\thispagestyle{empty} 

%%% Titelseite A
\vspace*{2\baselineskip}

\begin{center}
\sffamily
Universität Leipzig\\
Softwaretechnik-Praktikum\\
Sommersemester 2014
\vskip3\baselineskip

\bgroup
\Huge\textbf{Risikoanalyse}
\egroup
\vskip3\baselineskip

\begin{tabular}{ll}
Projekt & Graphical SPARQL Builder \\
Gruppe & s14.swp.gsb \\
Verantwortlich & \verantwortlicher\\
Erstellt am & \today \\
\end{tabular}

\vskip8\baselineskip

\bgroup
\sf
%\begin{table}[!h]
\begin{tabular}{lll}
\toprule
\textbf{Aufgabe} & \textbf{Verantwortlicher} & \textbf{eMail} \\\midrule

Projektleiter & Lukas Eipert & \url{mai13acl@studserv.uni-leipzig.de}\\
Technischer Assistent & Christoph Schultz & \url{mam08cxe@studserv.uni-leipzig.de}\\
Recherche & Elvira Akhtyamova & \url{mai13lmq@studserv.uni-leipzig.de}\\
Modellierung & Nico Graebling & \url{mai12kho@studserv.uni-leipzig.de}\\
Tests & Felix Helfer & \url{soz11gtd@studserv.uni-leipzig.de}\\
Implementierung & Nick Brückner & \url{mai12mdl@studserv.uni-leipzig.de}\\
Dokumentation & Siegfried Zötzsche & \url{che07bcj@studserv.uni-leipzig.de}\\
Qualitätssicherung & Felix Helfer & \url{soz11gtd@studserv.uni-leipzig.de}\\
\bottomrule
\end{tabular}
%\end{table}
\egroup
\end{center}
%%% Titelseite E

\pagebreak

%%%%%%%%%%%%%%%%%%%%%%%%%%%%%%%%%%%%%%%%%%%%%%%%%%%%%%%%%%%%%%%%%%%%%%%%%%%%%%%%
%\section{Einleitung}

%%%%%%%%%%%%%%%%%%%%%%%%%%%%%%%%%%%%%%%%%%%%%%%%%%%%%%%%%%%%%%%%%%%%%%%%%%%%%%%%
\section{Identifizierte Risiken}

\subsection{Zeitmanagement}\label{r1} 
  Bei einem Projekt dieser Größe spielt das Zeitmanagement eine zentrale Rolle für einen erfolgreichen Abschluss und birgt damit auch ein essentielles Risiko. 
  Besonders die mangelnde praktische Erfahrung der meisten Gruppenmitglieder diesbezüglich kann problematische Auswirkungen haben: Mangelnde Selbstorganisation einerseits, sowie\hack{\break} Fehleinschätzungen des benötigten Zeitaufwandes für einzelne Teilaufgaben andererseits, könnten eine Nichteinhaltung der gegebenen Fristen zur Folge haben. 
  Weitere denkbare Auswirkungen wären eine deutliche Kostenerhöhung in Form steigenden Zeitaufwandes und eine dem Zeitdruck geschuldete Qualitätsminderung. 
  Hinzu kommt durch studentische bzw. berufstätige Pflichten eine zusätzliche äußere Belastung, welche möglicherweise mit der Projektarbeit kollidieren und die Ressource Zeit weiter beschneiden könnte.\Hack{\looseness-1}
  

\subsection{Kommunikation intern}\label{r2}
  Fehlende interne Kommunikation, beziehungsweise mangelhafte Absprachen können das Klima innerhalb der Gruppe negativ beeinträchtigen. Unzureichende Präzision bei der Formulierung von Aufgaben und Anforderungen führen schnell zu Missverständnissen, entsprechend fehlerhafter Ausführung und damit steigender Belastungen. Auch eine zu ausführliche, redundante Besprechung kann negative Folgen haben, etwa wenn einzelne Mitglieder nicht erreichbar sind, oder über verschiedene Kanäle mehrfach zwischen unterschiedlichen Personen kommuniziert werden muss. All dies kann neben einem Mehr\-auf\-wand vor allem auch zu Spannungen und Frustration in der Gruppe führen, was wiederum möglicherweise das Projekt selbst gefährdet.\Hack{\looseness-1}
  


\subsection{Kommunikation extern}\label{r3}
  Auch die externe Kommunikation, also im wesentlichen der Austausch mit dem Auftraggeber, kann, wenn unzureichend durchgeführt, zu Problemen führen. Durch unklar formulierte Anforderungen oder nachträgliche Änderungswünsche seitens des Auftraggebers, beziehungsweise unpräzise Lösungsansätze seitens der Projektgruppe, kann es zu Missverständnissen kommen. Daraus könnten ultimativ Verzögerungen des Produkts resultieren, sowie eine Unzufriedenheit des Kunden, der seine Wünsche als nicht gewürdigt ansieht.\Hack{\looseness-1}

\Hack{\enlargethispage{1.0\baselineskip}}
\subsection{Ungünstige Rollenverteilung/Aufgabenverteilung}\label{r5}
  Die Organisation der Projektgruppe in Rollen soll für klare Aufgabenfelder und optimal verteilte Veranwortungsbereiche sorgen. Sie kann jedoch auch Risiken mit sich bringen: Untereinander sind die verschiedenen Rollen, besonders in ihrem inhärenten Aufwand, nicht ohne weiteres vergleichbar. Dies kann zu einer ungerechten Arbeitsverteilung führen. Außerdem entspricht auch die persönliche Präferenz und Zuteilung einer Rolle nicht immer den eigenen Fähigkeiten und Vorlieben, kann sich also auch erst im späteren Verlauf als unpassend herausstellen. Besonders die geringe Erfahrung im Bezug auf das Arbeiten innerhalb eines größeren Entwicklungsteams trägt erheblich zu diesem Risiko bei. Aus beiden Problemfelder resultieren dann möglicherweise Unzufriedenheit innerhalb der Gruppe und Mängel in der Qualität und der Einhaltung des Zeitplans.

\subsection{Personal}\label{r4}
  Sollte ein Mitglied für längere Zeit nicht am Projekt mitarbeiten können (z.\,B. aus gesundheitlichen Gründen) oder ganz ausscheiden (z.\,B. Studienabbruch/-wechsel) erhöht sich die Arbeitsbelastung der übrigen Gruppe.
  Zudem muss die Rolle innerhalb der Gruppe neu vergeben werden.
  Dadurch entsteht zusätzlicher Aufwand um sich in das zusätzliche Aufgabengebiet einzuarbeiten.
  Weiterhin wird die geplante Aufgabenverteilung und der Zeitplan für das erledigen der Aufgaben hinfällig.
  Das Wegfallen eines Gruppenmitglieds wirkt sich negativ auf jeden Aspekt des Projekts aus (evtl. mit Ausnahme der Kommunikation).\Hack{\looseness-1}
  


\subsection{Organisation}\label{r6}
  Eine gute Organisation ist essentiell für das Gelingen eines Projekts diesen Umfangs. Ebenso können jedoch diesbezügliche Mängel, zum Beispiel bei der Koordination der einzelnen Mitglieder, der zeitlichen Einteilung und der Dokumentation einen erfolgreichen Abschluss negativ beeinträchtigen. Auch können falsch gesetzte oder eingeteilte Prioritäten die Bewältigung wichtiger Teilaufgaben verzögern. Ein erheblicher Mehraufwand, Qualitätsminderungen und auch die Möglichkeit, dass Arbeiten unnötigerweise mehrfach ausgeführt werden, sind alles denkbare Folgen. 
  


\subsection{Erfahrung mit der Technik}\label{r7}
  Für die Mehrheit der Gruppe ist ein Großteil der Technologien sowie das gemeinsame Arbeiten an einem Softwareprojekt neu. Der Umfang des {\glqq}Neu-zu-Erlernenden{\grqq} wird den des {\glqq}Altbekannten{\grqq} deutlich übersteigen.
  Das Einarbeiten in Software und Konzepte sowie das Entwickeln und Testen von Arbeitsabläufen wird also einen erheblichen Teil des Zeitaufwands einnehmen.
  Es werden Anfängerfehler auftreten, welche im schlimmsten Fall gar nicht, oder erst spät erkannt werden.
  Daraus können ineffizientes (unnötig langsames) Arbeiten, qualitativ mangelhafter Code und eine ungünstige Wahl der zu verwendenden Technologien resultieren.
  


\subsection{Missverstehen der Aufgabenstellung}\label{r8}
  Ein Missverständnis der Aufgabenstellung bzw. der Wünsche und Vorstellungen des Auftraggebers kann zu verminderter Qualität des Produkts und im schlimmsten Fall zu einem falschen Produkt führen. Folgen sind unzweckmäßig investierte Arbeitszeit und ein zusätzlicher Aufwand durch Nacharbeiten.


\subsection{Technische Ausfälle}\label{r9}
  Technisches Versagen kann zum Teil- oder Komplettverlust der bis dato geleisteten Arbeit führen. Auswirkungen sind Verzögerungen und ein Mehraufwand durch Nacharbeitungen.

\Hack{\enlargethispage{1.0\baselineskip}}
\subsection{Mangelhafte Umsetzung: {\glqq}Handwerkliche Fehler{\grqq}}\label{r10}
  Eine mangelhafte Umsetzung des Projekts in technischer Hinsicht kann ineffizienten oder unlesbaren Code, Sicherheitsmängel, sowie ein fehlerhaftes Produkt als Resultat haben. Als Ursache hierfür wären sicherlich besonders die geringen Erfahrungswerte der Gruppe insgesamt zu nennen. Als Folge kommt es zu Qualitätsmängeln und einem erhöhten Zeitaufwand bei der Fehlersuche.


%%%%%%%%%%%%%%%%%%%%%%%%%%%%%%%%%%%%%%%%%%%%%%%%%%%%%%%%%%%%%%%%%%%%%%%%%%%%%%%%
\section{Maßnahmenplan}

\subsection{Zeitmanagement}\label{m1} 
  Eine der wichtigsten Maßnahmen ist hier sicherlich eine sehr exakte, kontinuierliche Aufwandserfassung, die sowohl den einzelnen Mitgliedern, als auch der Gruppe als Ganzes mit dem persönlichen und gesamtheitlichen Zeitmanagement helfen soll.
  Zudem soll versucht werden, Erfahrungen anderer Gruppen aktiv mit einfließen zu lassen und mit eigenen Wochenrückblicken während der Besprechungen die innerhalb des Projekts ge\-won\-nenen Erfahrungen zu reflektieren.


\subsection{Kommunikation intern}\label{m2}
  Hier sind vor allem die wöchentlichen Treffen und deren Protokollierung, in denen direkt wichtige Punkte besprochen und das weitere Vorgehen geplant werden kann, vorgesehen. Außerdem essentiell ist die Nutzung geeigneter, unkomplizierter Kommunikationsformen, wie dem Wiki und dem Forum des OLAT-Portals, sowie Email und Skype. Außerdem soll durch eine regelmäßige Evaluation der internen Kommunikation im weiteren Projektverlauf sichergestellt werden, dass der angestrebte Standard erhalten bleibt.


\subsection{Kommunikation extern}\label{m3}
  Einen möglichst reibungslosen Austausch soll vor allem das regelmäßige, protokollierte Treffen mit dem Auftraggeber ermöglichen. Dort können Vorstellungen und Ansätze direkt besprochen und erarbeitete Lösungsansätze vorgestellt werden. Dies gibt beiden Seiten die Chance, Rückfragen zu stellen und offene Punkte zu klären. Eine weitere Maßnahme könnte die Erstellung und Nutzung kleinerer Umfragen sein.


%\Hack{\enlargethispage{1.5\baselineskip}}

\subsection{Ungünstige Rollenverteilung/Aufgabenverteilung}\label{m5}
  Eine Umverteilung der einzelnen Rollen muss im Bedarfsfall möglich sein, falls sich herausstellen sollte, dass einzelne Positionen falsch, oder unverhältnismäßig besetzt wurden. Noch wichtiger ist eine gegenseitige Unterstützung in der Gruppe: Niemand sollte seine Aufgaben allein bewältigen müssen, wenn er sich dazu nicht ausdrücklich in der Lage fühlt. Stattdessen sollte eine gemeinsame, gerecht verteilte Bearbeitung des Arbeitspensums das Ziel sein.

\Hack{\enlargethispage{1.0\baselineskip}}
\subsection{Personal}\label{m4}
  Die Auswirkungen des Ausscheidens eines Gruppenmitglieds sollten so gering wie möglich gehalten werden.
  Daher ist es wünschenswert, das möglichst viele Gruppenmitglieder einen möglichst großen Einblick in jeden Teilaspekt des Projekts haben. Dabei muss nicht jeder alles wissen oder verstehen, es sollte aber möglich sein sich im Zweifelsfall in vertretbarer Zeit in jeden beliebigen Teilaspekt einarbeiten zu können.
  Das bedingt eine konsequente Dokumentation der erbrachten Teilleistungen.
  Weiterhin soll das wöchentliche Treffen und der Wochenrückblick genutzt werden um der ganzen Gruppe einen kleinen Einblick in das Erarbeitete zu bieten.
  Im Extremfall kann es passieren, dass das Projekt nicht im geplanten Umfang realisierbar ist. In dem Fall ist eine Anpassung der Projektzielsetzung erforderlich.




\subsection{Organisation}\label{m6}
\bgroup
\hack{\spaceskip\fontdimen2\font plus\fontdimen3\font minus2pt}
  Die hier angestrebten Maßnahmen überschneiden sich mit denen anderer Risiken, da der Überbegriff der Organisation natürlich auch einen übergeordneten Aspekt des Gesamtprojekts darstellt. So sollen vor allem die regelmäßigen Treffen der Gruppe, geeignete Kom\-mu\-ni\-ka\-tions\-ka\-nä\-le, welche ein gemeinsames Kommunikationsforum ermöglichen, und die Nutzung des Wikis als Wissensdatenbank für einen reibungslosen Ablauf sorgen. Mithilfe einer leicht überblickbaren Dokumentation durchgeführter und geplanter Arbeitsschritte und der frequenten Abstimmung untereinander soll eine effiziente Arbeitsumgebung geschaffen werden.
\egroup

\subsection{Erfahrung mit der Technik}\label{m7}
  Um Fehler früh zu erkennen und damit nicht alle jeden Fehler selbst noch einmal machen müssen, sollen die bereits gemachten Erfahrungen und das bereits vorhandene Wissen mit der Gruppe geteilt werden.
  Dabei sollen sowohl aufgetretene Fehler und Probleme als auch gefundene Lösungen vorgestellt werden und im Wiki festgehalten werden.
  Für die Recherche soll großzügig Zeit eingeplant werden um eine möglichst fundierte Wahl der Technologien zu gewährleisten. Es soll möglichst verhindert werden, dass eine zeitintensive Einarbeitung in eine Technologie stattfindet, die sich später als ungeeignet für das Projekt erweist.
  Weiterhin werden wir uns frühzeitig auf Standards einigen, beispielsweise für die Form des Codes um das gemeinsame Arbeiten am Code und das Verstehen desselben zu erleichtern.


\subsection{Missverstehen der Aufgabenstellung}\label{m8}
  Die Auftragnehmer formulieren schriftlich wie sie die Vorstellungen und Anforderungen der Auftraggeber verstanden haben und vor allem auch welche Aspekte unklar bzw. mehrdeutig sind und legen die Ergebnisse dem Auftraggeber vor.
  Dieser Schritt ist mehrfach im Verlauf des Projekts notwendig und sollte für die einzelnen Teilaspekte des Projekts wiederholt werden.
  Zusätzlich sollen frühzeitig Prototypen erstellt werden und in der Gruppe sowie mit dem Auftraggeber evaluiert werden.
  In der Umsetzungs- sowie Implementierungsphase setzen sich die Gruppenmitglieder zunächst einzeln, danach zusammen mit dem Problem auseinander.


\subsection{Technische Ausfälle}\label{m9}
  Die Daten sollen redundant vorliegen. Backups werden regelmäßig angelegt. Die Anbieter zur Datenvorhaltung werden nach Zuverlässigkeit ausgewählt.

\subsection{Mangelhafte Umsetzung: {\glqq}Handwerkliche Fehler{\grqq}}\label{m10}
  Neben der Dokumentation des Codes während der Implementierung sollen auch Tests und Code-Reviews nach der Hauptumsetzungsphase zur Qualitätssicherung herangezogen werden.

%%%%%%%%%%%%%%%%%%%%%%%%%%%%%%%%%%%%%%%%%%%%%%%%%%%%%%%%%%%%%%%%%%%%%%%%%%%%%%%%
% \section{Rollenverteilung}

% \vskip\baselineskip
% \bgroup
% \sf
% %\begin{table}[!h]
% \begin{tabular}{lll}
% \toprule
% \textbf{Aufgabe} & \textbf{Verantwortlicher} & \textbf{eMail} \\\midrule

% Projektleiter & Lukas Eipert & \url{mai13acl@studserv.uni-leipzig.de}\\
% Technischer Assistent & Christoph Schultz & \url{mam08cxe@studserv.uni-leipzig.de}\\
% Recherche & Elvira Akhtyamova & \url{mai13lmq@studserv.uni-leipzig.de}\\
% Modellierung & Nico Graebling & \url{mai12kho@studserv.uni-leipzig.de}\\
% Tests & Felix Helfer & \url{soz11gtd@studserv.uni-leipzig.de}\\
% Implementierung & Nick Brückner & \url{mai12mdl@studserv.uni-leipzig.de}\\
% Dokumentation & Siegfried Zötzsche & \url{che07bcj@studserv.uni-leipzig.de}\\
% Qualitätssicherung & Felix Helfer & \url{soz11gtd@studserv.uni-leipzig.de}\\
% \bottomrule
% \end{tabular}
% %\end{table}
% \egroup

%%%%%%%%%%%%%%%%%%%%%%%%%%%%%%%%%%%%%%%%%%%%%%%%%%%%%%%%%%%%%%%%%%%%%%%%%%%%%%%%
%\Hack{\pagebreak}

% \section{Abkürzungsverzeichnis}

% \begin{abk}

% \item[RDF] 
%   Resource Description Framework, eine formale Sprache zur Bereitstellung von Metadaten im WWW

% \item[SPARQL]
%   \textbf{S}PARQL \textbf{P}rotocol \textbf{A}nd \textbf{R}DF \textbf{Q}uery \textbf{L}anguage

% \end{abk}

%%% Local Variables: 
%%% mode: latex
%%% TeX-master: "main"
%%% End: 
