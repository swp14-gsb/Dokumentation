\section{Einleitung}

Dieses Handbuch richtet sich an Administratoren einer GSB-Instanz.
Dieses Dokument befasst sich mit den möglichen bzw. notwendigen Modifikationen um um den GSB für einen bestimmten SPARQL-Endpoint zu konfigurieren.

\section{config.js -- die Konfigurationsdatei des GSB}

Die Konfiguration des GSB zur Anpassung an einen bestimmten
SPARQL-Endpoint erfolgt in der Kondigurationsdatei unter
\verb+/app/js/config.js+.
Im Folgenden sind die Teile der Konfigurationsdatei aufgeführt, die
durch Administratoren angepasst werden können.

\subsection{propertyTypeURIs}

Die Liste der \verb+propertyTypeURIs+ legt fest welche Typen von
Eigenschaften dem Benutzer Angeboten werden.

Diese könnte wie folgt aussehen:

\begin{Verbatim}
propertyTypeURIs : {
  'OBJECT_PROPERTY' : [
    /http:\/\/dbpedia.org\/ontology\//  // Teil-URI als regulärer Ausdruck
  ],
  'NUMBER_PROPERTY' : [
    /http:\/\/www\.w3\.org\/2001\/XMLSchema#(integer|float|double)/,
    'http://www.w3.org/2001/XMLSchema#decimal'      // Teil-URI als String
  ],
  'STRING_PROPERTY' : [
    'http://www.w3.org/2001/XMLSchema#string',
    'http://www.w3.org/2001/XMLSchema#literal'
  ]
},
\end{Verbatim}
Die Liste besteht aus Eigenschafts-Typen wie \verb+OBJECT_PROPERTY+
oder \verb+NUMBER_PROPERTY+, die wiederum Listen von hinreichend
langen Teil-URIs der jeweiligen Eigenschaftstypen darstellen.
Diese Teil-URIs können einzeln als String oder durch einen Regulären
Ausdruck angegeben werden.

\subsubsection{OBJECT\_PROPERTY}

Unter \verb+OBJECT_PROPERTY+ wird die URI der zu verwendenden
Ontologie angegeben. Im Falle der dbPedia ist diese
\url{http://http://dbpedia.org/ontology/}, eine Entität vom Typ
\verb+Ontology+ (\url{http://www.w3.org/2002/07/owl#Ontology}).

\subsubsection{NUMBER\_PROPERTY}

Hier werden die URIs der Eigenschaftstypen aufgelistet, die eine
Arithmetik erlauben.

\subsubsection{STRING\_PROPERTY}

Hier werden die URIs der Eigenschaftstypen aufgelistet, die
String-Vergleiche erlauben.

\subsection{URLs}

\subsubsection{resultURL}

Der Rumpf der URL über die die vom Nutzer erstellte SPARQL-Anfrage an
den SPARQL-Endpoint gestellt wird, beispielsweise:

{\small\url{http://dbpedia-live.openlinksw.com/sparql?default-graph-uri=http%3A%2F%2Fdbpedia.org&format=text%2Fhtml&timeout=5000&debug=on&query=}}\medskip

Dabei ist
\begin{itemize}
\item \url{http://dbpedia-live.openlinksw.com/} die Adresse des Virtuoso-Endpoints
\item \url{sparql?} der Indikator, das eine SPARQL-Anfrage folgt
\item \url{default-graph-uri=http%3A%2F%2Fdbpedia.org} 
die URI des default-Graphen
\item \url{format=text%2Fhtml} 
das Format der Antwort. Dieses sollte bei text/html belassen werden um
eine für den Nutzer lesbare Form zu garantieren.
\item \url{timeout=5000} Das Timeout in ms für die Ausführung der
  Anfrage
\item \url{query=} der Indikator, dass die eigentliche Anfrage
  folgt. Dieser Teil \textbf{muss der letzte} der \verb+resultURL+ sein.
\end{itemize}

\subsubsection{queryURL}

Der Rumpf der URL über die der GSB die verfügbaren Klassen und deren
Eigenschaften beim SPARQL-Endpoint erfragt.
Das Format ist größtenteils analog dem der \verb+resultURL+ \textbf{jedoch
muss als Antwortformat json (\url{format=json}) angegeben werden}.

\subsection{allowedLanguages}

Voreinstellung: \verb#allowedLanguages : ['*','de','en','pl']#\smallskip

\noindent Wird einem Subjekt eine String-Eigenschaft zugewiesen, so
kann der Benutzer gewisse Anforderungen an diese Eigenschaft stellen.
Eine dieser Anforderungen ist die Sprache des Eintrags.
In dem Array der \verb#allowedLanguages# wird festgelegt, welche
Wahlmöglichkeiten bezüglich der Sprache dem Benutzer angeboten
werden. Die Einstellung \verb#allowedLanguages : ['en']# gibt dem
Benutzer beispielsweise nur die Möglichkeit nach englischen
String-Eigenschaften zu suchen.


\subsection{standardLang}

Voreinstellung: \verb#standardLang: 'en'#\smallskip

\noindent Wenn der GSB die verfügbaren Eigenschaften zu einem Subjekt bei einem
SPARQL-Endpoint anfragt lässt er sich ebenfalls den Kommentar
(\verb+rdfs:comment+) sowie das Label (\verb+rdfs:label+) zu den
jeweiligen Eigenschaften liefern, sofern diese in der als
\verb#standardLang# festgelegten Sprache verfügbar sind.

%%% Local Variables: 
%%% mode: latex
%%% TeX-master: "main"
%%% End: 
